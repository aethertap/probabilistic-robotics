
\documentclass[10pt]{article}
\usepackage{amsmath}
\begin{document}
\title{Solutions for Chapter 3 of Probabilistic Robotics}

\section{Problem 1}
\begin{enumerate}
  \item In this and the following exercises, you are asked to design a Kalman
filter for a simple dynamical system: a car with linear dynamics moving in a
linear environment. Assume $\delta t = 1$ for simplicity. The position of the
car at time $t$ is given by $x_t$. Its velocity is given by $\dot{x}_t$, and its
acceleration is given by $\ddot{x}_t$. Suppose the acceleration is set
randomly at each point in time, according to a Gaussian with zero mean and
covariance $\sigma^2 = 1$.

  \begin{enumerate}
  \item \textit{What is a minimal state vector for the Kalman filter (so that the
    resulting system is Markovian)?} 

    In order to be Markovian, we have to have a state vector such that the
    future and the past are independent given the present state. If the state
    vector has the position and velocity, this condition is met. All of the
    acceleration values of the past are completely summarized in the position
    and velocity, so keeping it as a state variable doesn't tell us anything new
    in terms of predicting the future. The acceleration is set randomly at each
    time step, but given that we know the position and velocity we don't need to
    know any past acceleration in order to compute the future given the state
    and the control (which in this case will be setting the acceleration).

    Our state vector is thus:
  $$\begin{pmatrix}x_t \\ \dot{x}_t\end{pmatrix}$$

  \item \textit{For your state vector, design the state transition probability 
    $p(x_t | u_t,x_{t-1})$. Hint: this transition function will possess linear
  matrices $A$ and $B$ and a noise covariance $R$.}

  We will use the moments parameterization since we're building a Kalman filter.
  The state transition function is of the form
  $$x_t = A x_{t-1} + B u_t + \epsilon_t$$

  Where $\epsilon_t$ is a gaussian random variable with mean zero (because the
  specified mean of 1 will be part of the control update) and variance 1 (as
  given in the problem).
 
  The mean for our distribution is given by $A x_{t-1} + B u_t$, and the
  variance is given by $A\Sigma_{t-1}A^T + cov(\epsilon_t)$. The
  $A\Sigma_{t-1}A^T$ term carries forward the covariance from the previous state
  updated by the state transition function. The $cov(\epsilon_t)$ term is the
  new error added in by the control action.

  All that remains is to use the equations of motion to derive matrices for $A$
  and $B$. We know that:
  \begin{gather}
    x_t = x_{t-1} + \dot{x}_{t-1} \Delta t \\ 
    \dot{x}_t = \dot{x}_{t-1} + \ddot{x}_{t-1} \Delta t
  \end{gather}

  We're going to ignore the acceleration term ($\ddot{x}_{t-1}\Delta t$) here
  though because we don't have that available in the state vector - it's part of
  the control update. The control update will update the velocity based on the
  acceleration value:

  \begin{gather}
    \Delta\dot{x}_t = \ddot{x}_t\Delta t 
  \end{gather}

  Which makes our control term look like this:
$$B_t u_t = \begin{pmatrix} 0 \\ 1\end{pmatrix} \left (1 \Delta t\right)$$

  Now, our prediction of the overall mean, in matrix form:

  $$
\overline{\mu}_t = \begin{pmatrix}x_t \\ \dot{x}_t\end{pmatrix} = 
\begin{pmatrix}1 & \Delta t \\ 0 & 1\end{pmatrix} \begin{pmatrix}x_{t-1} \\
\dot{x}_{t-1} \end{pmatrix}+
\begin{pmatrix}0 \\ 1\end{pmatrix} \Delta t
  $$

  Next we need to model the covariance. Since we are given the variance of the
  acceleration (=1), we have to figure out how that maps into variance of the
  two state variables and make a variance vector accordingly. Acceleration
  doesn't directly affect the position, but it does affect the velocity. Our
  state variance vector then looks like this:

  $$
\begin{pmatrix}\sigma_x^2 \\ \sigma_{\dot{x}}^2\end{pmatrix} = 
\begin{pmatrix}0 \\ \Delta t\end{pmatrix}
  $$

  With that variance vector, we can calculate the covariance as
  $\vec{v}\vec{v}^T$:
  $$
\begin{pmatrix}0 \\ \Delta t\end{pmatrix}\begin{pmatrix}0 & \Delta
t\end{pmatrix} = 
\begin{pmatrix}0 & 0 \\ 0 & \Delta t^2\end{pmatrix}
$$

  \end{enumerate}

\end{enumerate}

\end{document}
